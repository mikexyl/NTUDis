%=== FRONT PART ===
%=== ABSTRCT ===

%\begin{center}
\chapter*{Abstract}
%\end{center}
\addcontentsline{toc}{chapter}{Abstract}

SLAM (Simultaneous Localization and Mapping) is an important component of robot systems, enabling robot systems to exploit and map an unknown environment, while estimating its own position. Single robot SLAM techniques based on monocular or stereo cameras, color or depth images and segmentation- or feature-based algorithms, have been considerable developed with many successful solutions proposed, during years of research. Compared to single robot SLAM, multi robot SLAM have several advantage e.g. robustness and quicker exploration. However, the development of multi robot SLAM is much slower, because of the difficulties such as determination of relative poses of robots and communication, etc. CORB-SLAM as a multi robot SLAM system, of which the clients are based on ORB-SLAM2, inheriting its advantages like low-cost sensors and low computation cost, has not been evaluated in a quantitative method to assess its performance. For multi robot SLAM, the adaptability in life-long scenarios is another important research topic, since multi robots may run in different dates and seasons.
This work quantitatively evaluates CORB-SLAM, and experiments to combine it with illumination variance algorithm to improve its performance under different illumination and season.


\par
\textbf{Keywords:} Multi-robot SLAM, CORB-SLAM, Quantitative Trajectory Evaluation, Life-long SLAM.

%=== END OF CHAPTER ONE ===
